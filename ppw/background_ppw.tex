In the coming many-core era, due to the tight power budget, power efficiency is cricital for many-core processor design. In the previous work by Dong Hyuk Woo, evaluation of energy efficiency on the basis of performance and power (PPW) models is developed, which shows the tendency of PPW with number of cores. In this paper, We implement the dark silicon and thermal model into the evaluation of PPW to gain a better understanding of PPW in the dark silicon era.

In this section, we first present the Amdahl's law and performance-per-watt's traditional expression in multi-core processor. Then the static power modeling and thermal modeling is introduced, with which we can write the revised expression of performance-per-watt in multi-core processor.

\subsection{Amdahl's law}
Amdahl's law put an upper bound for the speedup that a multi-core processor can achieve by parallelization as:
\begin{equation}\label{speedup}
Perf = \frac{1} {(1-f)+f/n}
\end{equation}
where  $n$ is the number of cores, and $f$ is the fraction of a program's execution time that is parallelizable (0<$f$<1). Noted that, (1) is purely theroretical for it doesn't consider any constraints such as power budget.

The average power consumption of the many-core processor is as follows:
\begin{equation}\label{average_power}
W = \frac{P_{1} \times (1-f)+P_{n} \times f/n}{(1-f)+f/n}
\end{equation}
$P_{1}$ is the power consumption during the sequential computation phase, $P_{n}$ is the power consumption during the parallel computation phase.

$Perf/W$ of a many-core processor is expressed as:
\begin{equation}\label{ppw}
\begin{split}
\frac{Perf}{W} &= \frac{1}{(1-f)+f/n} \times \frac{(1-f)+f/n}{P_{1} \times (1-f)+P_{n} \times f/n}\\
&= \frac{1}{P_{1} \times (1-f)+P_{n} \times f/n}
\end{split}
\end{equation}

$P_{1}$ is composed of power consumption of 1 active core and $n-1$ idle cores, in previous works, due to lack of consideration of thermal constraints, $P_{1}$ is expressed as $1+(n-1)k$, in which $k$ stands for the fraction of power the processor consumes in idle state $(0 \le k \le 1)$. For parallel computation phase, $P_{n}$ consists of power consumption of $n$ active cores, which is previously defined as $n$.

\subsection{Static power modeling}

It is widely acknowledged that the total power of a chip is composed of dynamic and static power. The dynamic power is dependent on the activities of the chip, therefore it's easily estimated by methods implemented with performance counter. Yet the static power $p_{s}$ of the chip is caused by leakage current $I_{leak}$ as:
\begin{equation}\label{ps}
p_{s} = V_{dd}I_{leak}
\end{equation}
Due to the non-linear relationship between $I_{leak}$ and temperature, static power is also sensitive to temperature, which makes it hard to obtain. 

$I_{leak}$ is composed of many components, including subthreshold current, gate current, reverse-biased junction leakage current, et cetera. Among which, subthreshold current and gate current are the main parts of leakage current, therefore $I_{leak}$ can be approximated as:
\begin{equation}\label{I_leak}
I_{leak}=I_{sub}+I_{gate}
\end{equation}

Noted that $I_{gate}$ is cause by tunneling between the gate terminal and the other three terminals, does not depend on temperature and can be considered as a technology-dependent constant. Yet $I_{sub}$ is considered to be highly related to temperature, and can be modeled in the commonly accepeted MOSFET transitor model BSIM 4 as:
\begin{equation}\label{I_sub}
I_{sub}=Kv_{T}^{2}e^{\frac{V_{GS}-V_{th}} {\eta v_{T}}} (1-e^{\frac {-V_{DS}} {v_{T}} })\approx Kv_{T}^{2}e^{\frac{V_{GS}-V_{th}}{\eta v_{T}}}
\end{equation}

\subsection{Thermal modeling}
To estimate the power consumption of a IC chip, we first divide the chip and its package into multiple blocks called thermal nodes. Then, the thermal resistance and capacitance among these thermal nodes is computed. With above mentioned information, the thermal mdoel for a chip with $n$ total thermal nodes can be generated:
\begin{equation}\label{gt=bp}
\begin{split}
GT(t) + C\frac{dT(t)}{dt} &= BP_{T,t}\\
Y_{t} &= LT_{t}
\end{split}
\end{equation}

By applying thermal model, the effect of temperature on cores can be introduced. 
Through ergodic method, the distribution of light core with the maximum PPW
for number of light core from 1 to n can be specified.




