\section{Background}
In the coming many-core era, due to the tight power budget, power efficiency is cricital for many-core processor design. In the previous work by Dong Hyuk Woo, evaluation of energy efficiency on the basis of performance and power (PPW) models is developed, which shows the tendency of PPW with number of cores. In this paper, We implement the dark silicon and thermal model into the evaluation of PPW to gain a better understanding of PPW in the dark silicon era.

\subsection{Amdahl's law}
Amdahl's law put an upper bound for the speedup that a multi-core processor can achieve by parallelization as:
\begin{equation}\label{speedup}
Perf = \frac{1} {(1-f)+f/n}
\end{equation}
where  $n$ is the number of cores, and $f$ is the fraction of a program's execution time that is parallelizable (0<$f$<1). Noted that, (1) is purely theroretical for it doesn't consider any constraints such as power budget.

The average power consumption of the many-core processor is as follows:
\begin{equation}\label{average_power}
W = \frac{P_{1} \times (1-f)+P_{n} \times f/n}{(1-f)+f/n}
\end{equation}
$P_{1}$ is the power consumption during the sequential computation phase, $P_{n}$ is the power consumption during the parallel computation phase.

$Perf/W$ of a many-core processor is expressed as:
\begin{equation}\label{ppw}
\begin{split}
\frac{Perf}{W} &= \frac{1}{(1-f)+f/n} \times \frac{(1-f)+f/n}{P_{1} \times (1-f)+P_{n} \times f/n}\\
&= \frac{1}{P_{1} \times (1-f)+P_{n} \times f/n}
\end{split}
\end{equation}

$P_{1}$ is composed of power consumption of 1 active core and $n-1$ idle cores, in previous works, due to lack of consideration of thermal model, $P_{1}$ is expressed as $1+(n-1)k$, in which $k$ stands for the fraction of power the processor consumes in idle state $(0 \le k \le 1)$. For parallel computation phase, $P_{n}$ consists of power consumption of $n$ active cores, which is previously defined as $n$.

\subsection{Power consumption with thermal constraints}

It is widely acknowledged that the total power of a chip is composed of dynamic and static power. The dynamic power is dependent on the activities of the chip, therefore it's easily estimated by methods such as performance counter. Yet the static power $p_{s}$ of the chip is mainly affected by temperature, for it's caused by leakage current $I_{leak}$ as:
\begin{equation}\label{ps}
p_{s} = V_{dd}I_{leak}
\end{equation}

Due to the non-linear relationship between $I_{leak}$ and temperature, $p_{s}$ cannot be
calculated directly. The iteration method is traditionally implemented to solve such problems.

$P_{n}$ is made up of the power consumption of n active cores. Please note that in dark silicon
era, $P_{n} = nk$ is not the correct expression. Due to thermal constraint, Dynamic Voltage and
Frequency Scaling(DVFS) is necessary for Thermal Design Power(TDP) consideration. As a result,
the power consumption of each of the core may decrease with the increase of the number of active cores.
From xx, we have
\begin{equation}\label{gt=bp}
(G - B_{c}A_{s})T(t) + C\frac{dT(t)}{dt}= B_{c}(P_{d}(t) + P_{0})
\end{equation}

By applying thermal model, the effect of temperature on cores can be introduced. 
Through ergodic method, the distribution of light core with the maximum PPW
for number of light core from 1 to n can be specified.

In (2) and (3), $P_{1}$ is consisted of the power consumption of n-1 idle cores and 1 active 
core. The expression of $P_{1} = 1+(n-1)k$ is not implemented, for the power consumption of 
idle cores and active core is not consistant, due to the influence of the temperature.

By appending idle core model into the above mentioned thermal model, the distribution of idle
core and active core with the maximum PPW for number of light core from 1 to n can be specified.