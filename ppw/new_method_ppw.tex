\section{New Method}
In the coming many-core era, due to the tight power budget, power efficiency is cricital for
many-core processor design. In the previous work by Dong Hyuk Woo, evaluation of energy
efficiency on the basis of performance and power (PPW) models is developed, which shows the
tendency of PPW with number of cores. We implement the dark silicon and thermal model into
evaluation of PPW to gain a better understanding of PPW in the dark silicon era.
Different from previous work, the performance of the many-core processor is expressed with
the summary of operating frequency of all cores:
\begin{equation}\label{perf}
Perf = \sum_{i=1}^n f_{i}
\end{equation}

However, the $Perf$ is not able to take into account of fraction of computation that can be
parallelized, it should be replaced in further versions.
The average power consumption of the many-core processor is as follows:
\begin{equation}\label{average_power}
W = \frac{P_{1} \times (1-f)+P_{n} \times f/n}{(1-f)+f/n}
\end{equation}

$P_{1}$ is the power consumption during the sequential computation phase, $P_{n}$ is the power consumption during the parallel computation phase. 

$Perf/W$ of a many-core processor is expressed as
\begin{equation}\label{perf/w}
\frac{Perf}{W} = \sum_{i=1}^n f_{i} \times \frac{(1-f)+f/n}{P_{1} \times (1-f)+P_{n} \times f/n}
\end{equation}

\subsection{Power consumption of during sequential phase}

In (2) and (3), $P_{1}$ is consisted of the power consumption of n-1 idle cores and 1 active 
core. The expression of $P_{1} = 1+(n-1)k$ is not implemented, for the power consumption of 
idle cores and active core is not consistant, due to the influence of the temperature.

By applying thermal model, the effect of temperature on cores can be introduced. 