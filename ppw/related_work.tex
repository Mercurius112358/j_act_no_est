\section{RELATED WORK AND MOTIVATION}
\label{sec:related_works}
Power consumption has remained as one of the most important design constraints for many years. Typically, due to limited cooling capability and demand for higher performance, power budgeting and dynamic thermal management have become crucial.

Several techniques have been successfully applied for power budgeting in multi-core systems under thermal constraint. For instance, there are methods that employ dynamic voltage and frequency scaling (DVFS) to limit the system power consumption in order to keep the temperature under thermal constraints for both single-core system~\cite{Skadron:MICRO'03}, multi-core system~\cite{Jayaseelan:ICCAD'09}, and 3D IC~\cite{Coskun:DATE'09}. There are also researches that implement task migration to switch heavy and light tasks to meet the temperature constraint~\cite{Ge:DAC'10},~\cite{Chantem:TVLSI'11},~\cite{Liu:DATE'12}. DVFS and per-core power gating are exploited in~\cite{Lee:TVLSI'12} to achieve potential throughput improvement. The work presented in~\cite{Kultursay:CHSCSS'12} implements a reactive dynamic power partitioning 
algorithm to distribute the power budget unevenly among two CPU power domains to maximize instructions per second. %The above mentioned works are dedicated to improving performance of the system, the energy efficiency of them are not satisfactory.

The energy efficiency of multi-core IC systems has been studied by several researchers. Dynamic voltage scaling and dynamic core scaling are implemented in~\cite{Seo:TPDS'08} for energy saving. ~\cite{Majzoub:TCAD'10},~\cite{Kong:DATE'11} show how to determine the number of active cluter, task partition and frequency assignment for cluster-based multi-core system to reach higher energy efficiency. Methods to reduce the cost of data center operation by energy-efficient computing are presented in~\cite{Pedram:TCAD'12},~\cite{Wang:access'17}. Carrol and Heiser~\cite{Carroll:RTAS'14} investigate improving energy efficiency by frequency scaling and core offlining on mobile devices. Annamalai et al.~\cite{Annamalai:ISVLSI'14} describe a dynamic resource allocation and dynamic voltage and frequency adaptation based solution for determining the transient core speeds to maximize energy per instruction. Cochran et al.~\cite{Cochran:ICCAD'11} formulate a multinomial logistic regression model to estimate the optimal operating points. A hierarchical framework is proposed to address the problem of minimizing the energy consumption while maintaining a required throughput in~\cite{Ghasemazar:ISCAS'10}. Numerical analysis on energy efficiency of multi-core systems were studies in~\cite{Woo:Computer'08},~\cite{Hill:Computer'08},\cite{Sun:JPDC'10}. In~\cite{Hanumaiah:TCOMP'14}, techniques include DVFS, task migration and active cooling are implemented to achieve high energy efficiency for multi-core systems.

As discussed above, some of the limitations of the aforementioned works are as follows:
\begin{itemize}  
\item The dependence of leakage power to voltage and temperature is neglected;
\item The DVFS stages of core cannot be adjusted independently;
\item The thermal model does not distinguish the die and the package;
\item The thermal interaction among cores are neglected.
\end{itemize} 

In this work, we propose an energy-efficient dynamic power budgeting method to compute the power budget accurately and efficiently for dark silicon multi-core system, which can maximize the energy efficiency of the system. Our major contributions are summarized as follows:
\begin{itemize}
\item By analyzing the characteristics of PPW, we find that, for each active core number, we can generate a $\hat{T}_{opt}$, as long as the $\hat{T}_{opt}$ is applied to all active cores, maximum PPW of the system can be obtained, while the performance of the system varies with the active core distribution.
\item Based on the analysis above, we propose to formulate the power budgeting problem as minimizing the difference between $\hat{T}_{opt}$ and all cores' temperature, with greedy based acceleration, our method is capable of providing accurate power budget at runtime.
\item We show that our method can consider the transient temperature effect, by providing accurate power budget which automatically adapts to current running conditions of the system. We also show that our method can provide PPW boost when the system is at low temperature.
\end{itemize} 